\documentclass{beamer}
 
 
 \usepackage[utf8]{inputenc}
 \usepackage[T1]{fontenc}
 \usepackage[french]{babel}
 \usepackage{mathtools, bm}
 \usepackage{amssymb, bm}
 \usepackage{stmaryrd} 
 \usepackage{enumitem}
 \usepackage{xcolor}
 \usepackage{array}
 \usepackage{layout}
 \usepackage{pslatex}
 \usepackage{pstricks-add}
 \usepackage{lmodern}
 \usepackage{listings}
 \usepackage{color}
 \usepackage{pstricks,pst-node,pst-tree}
 \usepackage{tikz}
 \usepackage{mathrsfs} 

 % le thème artisanal :
 \usetheme{Warsaw}
 \usecolortheme{seahorse}
 \useoutertheme{infolines}
 \setbeamercolor{block body}{bg=normal text.bg!95!black!95!blue}
 \setbeamercolor{block body example}{bg=normal text.bg!95!black!95!green}
 \setbeamercolor{block title}{fg=white, bg=normal text.bg!50!blue!50!black}
 \setbeamercolor{block title example}{fg=white,bg=normal text.bg!50!green!50!black}
 \useinnertheme{rounded}
 \setbeamertemplate{blocks}[rounded][shadow=true]
 \beamertemplatenavigationsymbolsempty
 \renewcommand{\arraystretch}{1.5}


% blocs 
 \newcommand{\prop}[1]{\begin{block}{Proposition}#1\end{block}}
 \newcommand{\thm}[2]{\begin{block}{Théorème (#1)}#2\end{block}}
 \newcommand{\conj}[1]{\begin{exampleblock}{Conjecture}#1\end{exampleblock}}
 \newcommand{\defi}[1]{\begin{exampleblock}{Définition}#1\end{exampleblock}}
 \newcommand{\exmp}[1]{\begin{exampleblock}{Exemple}#1\end{exampleblock}}
 \newcommand{\diapo}[1]{\begin{frame}[allowframebreaks]#1\end{frame}}
 \newcommand{\dem}{\textbf{Démonstration }\hspace{0.5cm}}



 \newcommand{\Iff}[2]{\left[#1\,;\,#2\right]}
 \newcommand{\Ioo}[2]{\left]#1\,;\,#2\right[}
 \newcommand{\Ifo}[2]{\left[#1\,;\,#2\right[}
 \newcommand{\Iof}[2]{\left]#1\,;\,#2\right]}



 \newcommand{\pth}[1]{\left(#1\right)}
 \newcommand{\cro}[1]{\left[#1\right]}
 \newcommand{\acc}[1]{\left\{#1\right\}}
 \newcommand{\abs}[1]{\left|#1\right|}
 \newcommand{\dabs}[1]{\|#1\|}
 \newcommand{\scal}[1]{\left<#1\right>}
 \newcommand{\floor}[1]{\left\lfloor#1\right\rfloor}
 \newcommand{\ceil}[1]{\left\lceil#1\right\rceil}
 \newcommand{\dbcro}[1]{\left\llbracket#1\right\rrbracket}


 % pour faire des commentaires cool

 \newcommand{\esp}{\hspace{1cm}}
 \newcommand{\et}{\hspace{1cm}\text{et}\hspace{1cm}}
 \newcommand{\pet}{\hspace{0.5cm}\text{et}\hspace{0.5cm}}
 \newcommand{\ou}{\hspace{1cm}\text{ou}\hspace{1cm}}
 \newcommand{\pou}{\hspace{0.5cm}\text{ou}\hspace{0.5cm}}
 \newcommand{\avec}{\hspace{1cm}\text{avec}\hspace{1cm}}
 \newcommand{\comment}[1]{\hspace{1cm}\text{#1}\hspace{1cm}}
 \newcommand{\tq}{\hspace{0.25cm}/ \hspace{0.25cm}}
 \newcommand{\qt}[1]{(\,#1\,)\hspace{1cm}}
 \newcommand{\qti}[1]{\,,\hspace{1cm}#1}
 \newcommand{\vg}{,\,}

 \newcommand{\ssi}{\hspace{.2cm}\Leftrightarrow\hspace{.2cm}}
 \newcommand{\gssi}{\hspace{1cm}\Leftrightarrow\hspace{1cm}}


 % Pour la géométrie
 \newcommand{\vect}[1]{\overrightarrow{#1}}
 
 % divers 

 \newcommand{\mat}[1]{\begin{matrix} #1\end{matrix}}
 \newcommand{\pmat}[1]{\begin{pmatrix} #1\end{pmatrix}}
 \newcommand{\cotan}{\text{cotan}}
 \newcommand{\somme}[2]{\sum_{#1=0}^{#2}}
 \newcommand{\Vect}[1]{\text{vect}\pth{#1}}
 \newcommand{\ie}{\emph{ie} }
 \newcommand{\transp}{{}^t\!}
 \newcommand{\lleq}{<\!\!\!<}
 

 % pour les intégrales:
 \newcommand{\bigcro}[1]{\big[#1\big]}
 \newcommand{\de}{\,\text{d}}


 % Les ensembles:
 \newcommand{\Ce}{\mathbb{C}}
 \newcommand{\Er}{\mathbb{R}}
 \newcommand{\En}{\mathbb{N}}
 \newcommand{\Zed}{\mathbb{Z}}
 \newcommand{\Qu}{\mathbb{Q}}
 \newcommand{\Te}{\mathbb{T}}

 % En probas
 \newcommand{\prb}[1]{\mathbb{P}\pth{#1}}
\newcommand{\Esp}[1]{\mathbb{E}\cro{#1}}
 \newcommand{\Var}[1]{\text{Var}\pth{#1}}
\newcommand{\kt }{\,|\,}

\newcommand{\edistr }{\overset{d}{=}}
\newcommand{\cdistr }{\overset{d}{\to}}
\newcommand{\cproba}{\overset{\mathbb{P}}{\to}}
\newcommand{\easrly}{\overset{a.s.}{=}}
\newcommand{\casrly}{\overset{a.s.}{\to}}


 % spécifique

\newcommand{\dr}{\partial}


 \title[Mutations]{Estimer l’effet des mutations sur la fitness d’une population de bactéries}
 \date[2020-2021]{Encadrantes : Marie Doumic et Lydia Robert}
 \author[J. Andréoletti, N. Boutillon]{Jérémy Andréoletti et Nathanaël Boutillon}

\begin{document}


\begin{frame}
  \titlepage
\end{frame}

\begin{frame}
  \frametitle{Objectifs}
  \begin{itemize}[label=\bullet]
  \item estimer la loi des effets des mutations sur la valeur sélective (fitness);
  \item fitness mesurée par le taux de croissance des cellules.
  \end{itemize}

  \pause\vspace{0.5cm}
  On pose
  \[
  \begin{array}{r l l}
    s_i&=\frac{W_{t_{i-1}}-W_{t_i}}{W_{t_{i-1}}}&\comment{effet de la mutation $i$}\\
    N_t &&\comment{nombre de mutations avant le temps $t$}\\
    W_t &&\comment{fitness au temps $t$}
  \end{array}
  \]
d'où:
\[W_t=\prod_{i=1}^{N_t}(1-s_i)\]

$\to$ trouver la loi des $s_i$.
\end{frame}

\begin{frame}
  \frametitle{Experience $\mu$-MA (mutation accumulation)}
  \center\includegraphics[scale=0.3]{img/schéma_microMA.png}

  \flushright{\emph{Robert et al.}}
\end{frame}
\begin{frame}
  \frametitle{Experience MV (mutation visualization)}
  \center\includegraphics[scale=0.5]{img/schéma_MV.png}
  
  \flushright{\emph{Robert et al.}}
\end{frame}

\begin{frame}
  \frametitle{Résultats de l'article}
  \begin{itemize}[label=\bullet]
  \item estimation des premiers moments de la loi des effets des mutations (DFE);
  \item les mutations apparaissent selon une loi de Poisson;
  \item les effets des mutations sont indépendants;
  \item $1\%$ des mutations sont létales; $0,3\%$ des mutations sont fortement délétères; 
  \end{itemize}
\end{frame}

\begin{frame}
  \frametitle{Choses à faire}
  \textbf{Énoncé du problème:} estimer la loi des $s_i$ (qui sont iid) sachant que l'on sait estimer la loi des $W_t$ ($t\geqslant 0$), et sachant que \[W_t=\prod_{i=1}^{N_t}(1-s_i)\]
  \pause
  \begin{itemize}[label=\bullet]
  \item Estimer les moments de la loi des $s_i$ et en déduire une estimation de la loi des $s_i$;
  \item Faire un programme en Python pour tester les résultats que l'on obtiendra en partant d'une distribution fixée pour les $s_i$;
  %\item Pour l'étude des mutations les plus délétères, on peut supposer qu'elles sont rares et qu'il est possible d'évaluer leur effet directement en regardant l'évolution de la fitness qu'elles ont causées (ainsi, on suppose qu'une seule mutation fortement délétère est apparue lors de la division, et que l'effet de cette mutation est grand devant les effets des autres mutations);
  \item Utiliser le fait que $\ln{W_t}$ est une somme de variables aléatoires iid pour trouver la loi de ces variables.
  \end{itemize}
\end{frame}

  \begin{frame}
    \frametitle{Références}
\begin{thebibliography}{1}

\bibitem{rob}
  Robert et al.,
  \emph{Mutation dynamics and fitness effects followed in single cells}, Science 359, 1283–1286, 16 March 2018
  
\end{thebibliography}


\end{frame}





\end{document}
